%%==================================================
%% conclusion.tex for BIT Master Thesis
%% modified by Liu xiahua
%% version: 0.1
%% last update: May 25th, 2018
%%==================================================


\begin{conclusion}

电能动力系统中的主要组件是电动机和发电机、电池或能量存储系统以及功率电子驱动系统。但是相比于其他的电动车技术,电池技术是阻碍纯电动汽车在大众市场上发展的主要技术之一。本文对于现有的电动汽车领域的技术进行了回顾和总结,对于我国的电动汽车发展的现况和国外进行了对比,在电动汽车领域,目前我国汽车工业的技术与国外的汽车企业相比存在一定的差距。

在电池箱设计技术中,最首先要进行研究的,也是最有商业意义的就是电池箱参数的匹配,因为市面上的大多数混合动力车辆还是基于原有车辆的动力系统开发的,参数匹配在实际生产中有重要意义。本文提出了一种新型的混合动力汽车的电池箱参数匹配的方法,在给定的汽车的行驶动力性指标下,利用车辆的动力学方程,分别通过对于电池箱的功率和能量两部分参数计算,确定电池箱的参数。仿真的结果验证了本文理论计算正确和有效,也同时反映了电池参数匹配中,有许多因素均会影响参数匹配的结果,对于电池参数模型的确定要考虑更多因素。而在电池箱的机械设计方面,现有与电池设计有关的标准大多数出于对电池安全方面考虑。本文分别对于单体、模组、整体和机械连接部分进行了设计,考虑到电池箱的抗振,热管理方面,可以对于现有的设计进行完善和补充。

由于电池箱作为电动汽车的储能元件,是电动汽车中最重要的部件之一。而对于电池箱等电气系统的诊断在过去的十年内是由各个汽车企业自己定制的,不具备行业统一的标准,这也对电动汽车的维修带来了困难。本文中设计的 OBD 系统是基于标准 OBD 协议设计的,提高了诊断流程的统一性,规范故障的条件定义和故障码的定义,该系统的设计具有行业前瞻性意义,响应了我国政府的最新要求。本文在设计过程中对于多种硬件方案进行了分析,在最后进行了相关电路原理图以及 PCB 的设计。

总体来说,本文的设计跨越了电池系统设计的数方面,包含了动力参数、机械模型、电子线路、通信数个系统的设计。虽然设计范围较广,在设计时各因素不能够完全覆盖,但是本文提出的电池参数匹配模型和机械模型以及电子电路的设计,在仿真校验和实际测试中性能优良,具有参考的价值。在最后,本文对于电池匹配参数模型的优化以及热管理系统等后续设计提供了思路和方法,在未来的研究中可以进一步完善和改良。

\end{conclusion}