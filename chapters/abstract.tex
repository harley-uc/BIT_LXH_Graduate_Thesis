%%==================================================
%% abstract.tex for BIT Master Thesis
%% modified by yang yating
%% version: 0.1
%% last update: Dec 25th, 2016
%%==================================================

\begin{abstract}

电动车在汽车行业占有量逐渐上升,对于电池系统的设计成为了热点,为了探究电动汽车电池箱的设计流程,本文将分四个章节介绍电池箱设计中不同的方面。本文第一章节对于现有的电动汽车的技术现状做了一个回顾,并且对于本文涉及的混合动力汽车的分类进行了详细的介绍。第二章节依据车辆动力学原理,在实际给定的电池单体参数下,理论计算与现有的某车型车辆数据相匹配的电池箱参数,接着利用 AVL Cruise 车辆仿真软件建立了与之对应的串联式混合动力汽车的车辆运动参数模型,在 NEDC 工况下仿真验证所计算的电池箱参数的实际性能。在第三章节中,本文对于现有的电池箱机械结构的标准进行了总结和分析,并综合考虑减震,机械固定,热管理,以及热失控管理等方面因素,使用 SoildWorks 建模软件绘制了符合工业标准的,与第一章节中计算的参数一致的电池箱的机械结构。在第四章节中,对于现有的车载诊断系统进行了介绍,并针对当今电池箱的诊断系统进行了深入研究。结合实际例子,针对该电池系统设计了一种基于 OBD-II 的诊断系统,使其满足目前的国家轻型燃油车标准要求。

\keywords{插电式混合动力汽车;电池系统;机械设计;电池管理系统;车载诊断系统}
\end{abstract}

\begin{englishabstract}

    As electric vehicle sales escalate in the automotive market nowadays, research on battery systems has become a hot topic. In order to explore the design steps for electric car battery boxes, This article introduces the different aspects of battery box design in four chapters. The first chapter provides a review of the state of the art of electric vehicles, and gives a detailed introduction to the classification of hybrid cars involved in this article. In the second chapter, according to the principle of vehicle dynamics, this article theoretically calculates the battery box parameters that match the data of an existing vehicle model under the actual given battery cell parameters, then uses AVL Cruise vehicle simulation software to build a corresponding series electric car model for hybrid vehicles, and observes the virtual performance of the calculated battery box parameters under New European Driving Cycle working conditions. In the third chapter, this paper summarizes and analyzes the existing standards for the mechanical structure of battery boxes, then uses the SoildWorks modeling software to draw the mechanical structure of the battery box in the first chapter, which meets the requirements of industry standards including vibration absorption, mechanical fixation, heat management, and run-away control. The fourth chapter of this paper reviews the existing vehicle diagnostic system, and makes a thorough study of the battery box diagnostic system, moreover in this chapter a diagnostic system based on OBD-II is designed for the battery system to meet the requirements of the current national light fuel vehicle consumption standard.
    
 \englishkeywords{Plug-in hybrid vehicles; Battery System; Mechanical Design; Battery Management System; On Board Diagnostics}
 
 \end{englishabstract}