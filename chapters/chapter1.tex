%%==================================================
%% chapter01.tex for BIT Master Thesis
%% modified by yang yating
%% version: 0.1
%% last update: Dec 25th, 2016
%%==================================================
\chapter{绪论}
\label{chap:intro}
\section{本论文研究的目的和意义}

电动汽车,也被称之为电驱动汽车,使用一个或者多个电动机来提供动力,电动汽车的能量储存形式有各种类型,可以是使用自身车载电池来储存能量,也可以是使用外部的线路,如有轨电车,供给能量给驱动电机 \cite{phev-zongshu}。

电动汽车的能量储存系统通常是由化学电池组组成,在各种化学电池种类中,由于锂离子电池在循环寿命、单体比能量、放电倍率等方面均具有优良的性能表现 \cite{nengliangchucun},目前市面上存在的大多数电动汽车的电池单元都采用了锂离子电池。但是,锂离子电池的高昂的价格,和其有限的能量密度(与燃油相比)导致了仅搭载锂离子电池的纯电动车辆在拥有合适续航里程的条件下,电池箱重量较大。而混合动力汽车相比纯电动汽车,在一方面使用一个或多个电动机来驱动汽车,另一方面保留了一个排量较小的内燃机,通过连接发电机,产生电能补充电池系统的能量,或在较优的工况下直接使用发动机驱动车辆。混合动力汽车方案对于车载电池系统的要求较为宽松,需求储存能量要求较小,设计中不仅减小了电池组的体积和重量,节约了成本,在日常使用中,混合动力汽车电池系统的工作负担相对于纯电动汽车较小,同时增加了电池的使用寿命。

插电式混合动力汽车中的电池系统的能量输入方式有两种:可以使用外部充电接口,通过电网给电池充电,日常使用时,在较短的行驶里程中(如 40 千米之内),仅仅使用电池系统中的电能驱动车辆,车辆工作在纯电动模式。如果需要一次行驶较长路程,混合动力系统会运行内燃机,连接发电机给电池系统供电,或直接驱动车辆。插电式混合动力汽车结合了内燃机的高能量密度,可以使车辆的整体重量保持较轻,但相对于传统内燃机车辆在排放和经济性方面有较大优势;又结合了纯电动汽车的较高能量利用效率优势,但相比纯电动汽车,又具有行驶里程长,整车价格较低的优势。插电式混合动力汽车其电池系统与纯电动汽车不同,技术方面存在一些差异,目前各大整车制造商,譬如吉利,和一些系统供应商,如如苏州绿控和天津松正,都在开展插电式混合动力汽车的能量储存系统的技术方案的研究,并大力推广插电式混合动力汽车。

由于插电式混合动力汽车涉及到两部分的能源供给装置,分别为电池箱和内燃机。故驱动系统的拓扑结构设计,能量的合理分配和输出就显得尤为重要,因为这不仅仅关系到能量的使用效率以及车辆的经济性,还和车辆的动力性能有着重要联系。由于发动机的制造与加工技术因素,发动机的性能参数较难调整,与之相比,电池系统中电池单元具有模块化特点,设计时可以方便地增添电池单元来调整参数,所以在当今情况下,如何匹配电池系统的性能参数成为了混合动力汽车电池系统设计中的一大技术要点。本文前一段将会介绍几种目前常用的电池系统匹配算法,并结合实际的电池模组,对于一个实际给定的插电式混合动力车型进行电池系统的匹配。

汽车电池系统储存着巨大的能量,所以控制能量的释放过程,预防能量失控,对于电池系统至关重要。车载电池系统中的电池管理系统(Battery Management System 又被称作 BMS)起到了对于电池的维护,检测的功能,在 BMS 系统中 On-Board-Diagnosis (车载诊断系统,又被称作 OBD 系统) 是电池箱中 Battery Management System (电池管理系统,又被称作 BMS 系统) 中的一个重要组成部分。OBD 系统会在车辆运行或者停车时持续对 BMS 系统采集到的电池箱各项数据(譬如箱内温度,剩余电池容量,电池单体开路电压等)进行数据分析,实时检查故障并储存,将电池箱的健康状况反馈给驾驶员,内部存储的故障信息则可以通过一个标准化的 OBD 接口,使用外部专业的检测装置读取 \cite{鲁学柱2006OBD技术及其发展}。最新的 OBD 技术可以通过以太网将故障信息发送给相关的政府机构或者车辆制造商,以便维修和管理。美国是最早规定车辆必须装配车上诊断系统的国家,之后欧盟与日本也陆续采行。中国则自 2006 年起陆续对新车推出了装备车上诊断系统的要求。OBD 系统在美国的颁布实施,给汽车专业人士的诊断带来了空前的便利。在中国《国家第六阶段机动车污染物排放标准》(预计 2020 年实施)中,对于插电式混合动力汽车的电池系统故障和混合动力汽车中发动机的排放诊断做出了明确规定,所以对于插电式混合动力汽车电池 OBD 诊断新系统的研究,具有市场和技术上的重大意义。

\subsection{混合动力汽车的分类}

混合动力汽车按照其驱动链的拓扑结构可以分成以下几类:并联式混合动力汽车,串联式混合动力汽车,能量分流式混合动力汽车。不同种类的混合动力汽车的能量耦合类型不同。对于不同的混合动力驱动系统结构类型,均可使用制动回收功能来回收制动能量。

\begin{itemize}

	\item 并联式混合动力汽车

	      并联式混合动力汽车中,驱动能量在减速器或者车桥处耦合,输出至车轮,在动力分配方案方面相对于串联式混合动力较为灵活,可以使用算法来按照不同工况调整 “发动机-电机” 的输出功率比。由于发动机和电机两套驱动系统均可以单独驱动车轮,动力性较好,相较于串联式混合动力汽车,其内燃机输出能量利用效率较高,结构较为简单,但是独立的驱动系统同时也带来了一个缺点,当电池的电量耗尽时,车辆无法使用发动机为电池充电,只能使用发动机驱动车辆。由于车轮直接与内燃机连接,在本质上没有改变内燃机的排放特性,在路况较差的情况下,内燃机排放量会大大超过串联式混合动力汽车。

	      在市面上搭载此类驱动系统的车型较多,例如比亚迪秦等。

	\item 串联式混合动力汽车

	      串联式混合动力汽车是在原有的驱动链结构上增加了一个发电机和电动机 \cite{串联式混合动力电动汽车先导车的研究开发}。在这种结构中,发动机不直接驱动车轮,而是通过发电机给电池系统提供电能,电池系统给电动机供电,驱动车辆运行。串联式混合动力汽车技术核心集中在能量控制算法方面,通过控制发电机相连发动机的启停,保证电池系统的 SOC 在一定范围内,可以有效增加电池寿命,并确保汽车能够拥有足够的行驶里程。串联式混合动力车辆由于仅由电动机驱动,所以在动力性方面不如并联式混合动力汽车,但是由于发动机不直接连接车轮,可以保证其工况不受路面影响,可以保持较高的能量转换效率,缺点是其在高速行驶状态下,燃油利用效率相较于发动机直接驱动车轮较低。

	      市场上使用串联式混合动力结构的车型有:雪佛兰沃蓝达、宝马 i3 增程式混合动力汽车、传祺 GA5 增程式混合动力汽车。

	\item 能量分流式混合动力汽车

	      能量分流式混合动力汽车,又被称作混联式混合动力汽车,是由丰田的 “E-CVT” 结构为代表的较少车型采用的结构 \cite{2006E-CVTHYBRIDTRANSMISSION},汽车中的发动机通过一个装置,既可以选择给电池系统供电,也可以直接驱动车轮。这种结构综合了串联和并联结构的优点,汽车在低速状态下或较差的工况下使用电动机驱动方式,在车速较高的状态下使用发动机直接驱动车轮,使得车辆具有十分高的燃油经济性,但是由于其结构较为复杂,故车型较少,价格较高。

	      市场上使用能量分流式混合动力汽车的车型有丰田品牌下的:丰田卡罗拉双擎,丰田普锐斯双擎系列等。
\end{itemize}

\section{国内外研究现状及发展趋势}

目前,由于插电式混合动力汽车市场份额逐年增长,根据 Navigant Research 的最新报告显示在 2017 年中,全球的插电式混合动力汽车销售总额超过了一百万辆。由于市场的驱动,国内外各大汽车相关企业都在加大力度进行有关技术的投资,在混合动力汽车方面主要的研究方向有以下几种:


\subsection{高性能电池技术}

目前对于电池技术的大部分研究工作都是针对增加电池的能量密度,降低电池的成本,同时保持电池较高的循环寿命等方面。美国先进电池联盟 (USABC) 对于 2020 年电池组设定了目标 \cite{nengliangchucun}。目前已经商用的电池单体,主要分为以下几种类型:镍氢电池、钠-氯化镍电池、锂离子电池、锂基电池。每种化学物质不同的电池单体都通过 7 种不同的指标进行评估,并与 USABC 2020 目标进行比较,如表 \ref{tab:USABC} \ref{tab:LiUSABC} 所示,在不同类型的电池种类的比较中,锂离子电池在各项参数中具有相当的优势,但是考虑到各项参数,并没有一个完美的电池技术能够达到目标要求,所以发展电池技术是解决目前电动汽车诸多问题的关键途径。

\begin{table}
	\centering
	\caption{USABC 目标与现有的电池性能参数} \label{tab:USABC}
	\begin{tabular*}{0.9\textwidth}{@{\extracolsep{\fill}}cccccc}
		\toprule
		参数 & USABC 目标 & 铅酸电池 & 镍氢电池 & ZEBRA 电池\\
		\midrule
		比功率 (W/Kg)&700&75-150 &80-400&150-200 \\
		能量密度 (Wh/L)&750&50-80 &60-150 &135-180 \\
		比能量 (Wh/Kg)&350&30-50&45-80&100-120\\
		自放电 (\%/day)&0.03&0.29-0.57&1-1.43 &15\\\
		价格 (\$/kWh)&100&100-150 &150-250 &100-200\\
		循环寿命 (循环次数)&1000&500-1000 &1200-2000 &>2500\\
		热失控 (°C)&220&60-100 &110-175 & - \\
		\bottomrule
	\end{tabular*}
\end{table}

\begin{table}
	\centering
	\caption{各类锂离子电池特性} \label{tab:LiUSABC}
	\begin{tabular*}{0.9\textwidth}{@{\extracolsep{\fill}}cccccc}
		\toprule
		参数 & USABC 目标 & 铁锂电池 & 锂镍锰钴氧电池 & 三元锂电池 \\
		\midrule
		比功率 (W/Kg)&700&250-1600&500-2400&700-800\\
		能量密度 (Wh/L)&750&250-500&230-550&500-670\\
		比能量 (Wh/Kg)&350&80-140 &126-210&145-240\\
		自放电 (\%/day)&0.03&0.1-1.29&0.1-0.71&0.1-0.57\\
		价格 (\$/kWh)&100&300-600&300-600&300-600\\
		循环寿命 (循环次数)&1000&1000-2000&1200-1950&1000-1280\\
		热失控 (°C)&220&195&168&136-160\\
		\bottomrule
	\end{tabular*}
\end{table}

\subsubsection{锂离子电池技术}
电池技术在过去的几十年里发展十分迅速。锂离子电池技术过去一直作为智能手机和笔记本电脑中的大部分移动用途的电力储存载体,但在最近也被用于驱动车辆和智能电网用途。针对纯电动汽车来说,锂离子电池是能量密集型的,因此它允许车辆具有更长的纯电动行驶范围;然而,它们通常不具有提供或不能承受由车辆的动态功率分布引起的大功率尖峰。对于再生制动情景来说,锂离子电池可以处理的能量接收水平是有限的;因此,大部分再生能量都被机械制动器消散。在高动态功率分布情况下,锂离子电池可能在某些状态下大大超负荷工作,这对其使用寿命产生了负面影响。施加在电池组上的高动态负载曲线引起电池的退化,导致电池内部电阻增加,后果是电池容量衰减,导致电池过早失效。

锂离子电池电极的构造中使用了各种材料。阴极材料通常是锂汞齐的氧化物变体,其通常含有锰(LMO),钴(LCO),镍,磷酸铁(LFP)或其混合物,例如 LiNiMnCo(NMC)和 LiNiCoAlO2(NCA),和铝混
合物。阳极材料通常是石墨,虽然硬碳,硅—碳化合物,钛酸锂,锡或钴合金以及硅—碳混合物也已经用于消费电子产品的电池中。在上述所有类型中,锂离子在电极之间来回运动,并且在基于分子插入的反应中传递电子,而不是传统的分子与分子之间的化学反应。锂离子电池技术的优点是循环寿命更长,库仑效率高(高达 98%),低自放电率。可以使用各种各样的电解质材料,从固体基质到液体基质,通常是有机非水形式 \cite{nengliangchucun}。

轻质材料的使用和锂离子电子转移的高电压电势导致高功率 / 能量密度和较高的标称电压,例如 3.2-3.8V。由于其拥有较高的热失控温度,LiFePO(LFP)变体被认为是最安全的锂离子电池化学物质之一。同时,与其他技术相比,NMC 和 NCA 电池正在主导电动汽车市场。因此,像松下,特斯拉,LG 化学和三星 SDI 这样的公司正在大力投资这两种电池技术。目前,锂离子化学反应正在得到广泛的研究和开发,目标是显著提高能量储存能力以及工作电压。例如,NMC 阴极的改进使锂离子电池的工作电压从 4.13V 增加到 4.3V。目前已经在锂钒磷酸盐阴极中观察到高达 4.7-4.8V 的工作电压,这些阴极已经被集成到 Subaru 64e 原型中。尽管研究机构的一些成员正在关注传统锂离子化学品方面的不断改进,但也有其他许多人则认为下一代基于锂的电池可以显著超越传统的锂离子电池。

电池技术还包括电池管理系统技术。电池管理系统的主要功能有电池箱保护、估计电池状态、控制电池配置,以确保电池工作的安全,并且延长电池箱的使用寿命。

\paragraph*{电池箱保护:}电池箱内部存在被动的保护措施,在目前的技术条件下,通常使用正温度系数自恢复保险丝和使用双金属断路器的短路保护每个电池免受过电流的影响。并且在设计的阶段就应考虑到电池单体的隔热和泄压功能。但是这些措施并不能预防故障的出现,在电池箱内部出现异常,如出现高温或高压,或者外部预期之外的物理条件,如振动,冲击等情况出现时,电池管理系统应该主动阻止故障的出现。在工作过程中,电池管理系统必须分别监控每个电池,包括电池电芯的温度,内阻,和电池箱内部的温度,压力等参数,确保每一个参数都严格地控制在操作范围内。

由于电动汽车发展历史相对于燃油车相比较晚,故在燃油车上已经完善的 On Board Diagnosis System,又被称作车载诊断系统,对于电动汽车或混合动力汽车的故障诊断项目并没有作详细的定义,本文在后续章节对于如何在电池系统中应用车载诊断系统做了详细的设计,对于 OBD 系统的协议和故障码是如何定义的进行案例演示,希望对于未来在电动车系统上应用 OBD 系统提供一个参考。

\paragraph*{电池参数状态估计:}电池系统的能量管理,与电池的参数估计息息相关,如何精确地估计电池当前的能量状态,是目前电池管理技术研究的一个方向。通常的电池内部容量是以电池单体内的电荷量 Qn 表征的,SOC(State of Charge)电池电荷状态不能够直接测量得到,而是被映射到开路电压 OCV 中,由于电池的寿命因素和外部温度的变化,电池的 SOC-OCV 曲线通常是非线性函数,并且在不同的电池使用情况下不同。在目前的对于 SOC 的研究集中在如何构建一个符合电池特性的 SOC 模型当中 \cite{BatteryManagementSystemUsedinElectricVehicles}。

\paragraph*{能量的平衡:}如果电池组内部没有能量平衡,则其有效容量取决于电池组中能量最少的一块单体的能量。因此,现代 BMS 系统使用了平衡技术达到能量在不同的模块和单体之间的转移,平衡有两种类型:耗散型和非耗散型,耗散型平衡从 SOC 最高的电池中提取出能量,并将能量通过分流电阻或晶体管中耗散掉;非耗散型平衡将能量从较高 SOC 电池转移到较低 SOC 的电池中,这需要电池管理系统使用控制策略实现对于转移链路电流的控制 \cite{兰祥2017无人机锂电池组平衡充电技术的研究}。

\subsection{高性能电驱动技术}
电驱动技术的范围包括了电动机技术,电动车辆传动技术。在当今技术条件下,可以使用的电动机种类有:直流电动机,交流电动机,永磁同步电机,开关磁阻电机。由于直流电动机的有刷特性,不适合作为大电流,高转速的应用场景,而开关磁阻电机作为一种新技术,目前的发展并不成熟,仅有少量车型搭载该类型电机。市场上,大多数电动汽车配备的电机类型为交流电动机和永磁同步电机。

电动车辆的传动系统由于电动机的转矩转速特性与传统内燃机不同,而且对于混合动力汽车,配备有两个或以上的驱动系统,而且功率输出存在耦合,电驱动技术相比传统驱动技术更加复杂,如何提高传动效率,增加汽车的经济性,成为了当前电动车技术的关键点。

\subsection{国外发展状况}
在混合动力汽车领域,国外的企业在技术方面较为先进,日本的丰田汽车、本田汽车和美国的通用汽车拥有此领域的核心技术,如混合动力系统的总成以及控制算法,日本的丰田汽车在能量分流式混合动力汽车方面拥有 “E-CVT” 专利 ,其“双擎”混合动力汽车与在能量利用效率上较其他方案效率更高 \cite{2006E-CVTHYBRIDTRANSMISSION},国内目前的混合动力汽车主要有串联式和并联式两种类型,在混联式混合动力汽车方面与外国相比尚有差距。

在 OBD 车载诊断系统领域,由于欧洲和美国是 OBD 标准的制定国,在 OBD 技术方面发展较为先进。目前我国的汽车部件生产厂家中相关零部件的 OBD 系统,大多数要向外国的系统开发者支付费用购买现有的 OBD 系统和外围的测试硬件进行开发,且 OBD 系统售价昂贵。