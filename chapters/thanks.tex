%%==================================================
%% thanks.tex for BIT Master Thesis
%% modified by yang yating
%% version: 0.1
%% last update: Dec 25th, 2016
%%==================================================

\begin{thanks}

本论文的工作是在导师程夕明老师的指导下进行的,在论文的写作的初期我遇到了许多问题,但在老师的悉心帮助下全部都顺利解决。在进行参数匹配的过程当中,老师对于我的匹配计算过程进行了指导,指出了我在计算过程中的错误,并且提供了匹配中所需要的相关参数,在第四章中对于本文电池箱的诊断系统设计也提出了设计的建议。感谢北京理工大学电动车辆国家工程实验室,韩孟佐师兄不吝抽出时间做实验,为我的论文中的电池匹配章节提供了电池单体的内阻与 OCV-SOC 的实验详细信息。

另外感谢学校给我提供的优秀的学习环境,在大学四年当中我收货了相当丰富的知识,并且在这个过程当中积累了不少的工作经验,在无人车车队担任组长和在校外参与实习增长了我的经历和见识,使我不再对自己的未来迷茫。

最后,感谢北京理工大学车辆系的老师和同学,在北理工学习生活的日子有优秀的老师和同学相伴,我倍感荣幸。
\end{thanks}
